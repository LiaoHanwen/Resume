% !TEX program = xelatex
\documentclass{resume}
\usepackage{lastpage}
\usepackage{fancyhdr}
\usepackage{linespacing_fix}
\usepackage[fallback]{xeCJK}

\thispagestyle{empty}

\begin{document}

\renewcommand\headrulewidth{0pt}
\name{廖瀚文}

% fill space
\vspace{6pt}

\basicInfo{
    \email{hanwen@nyu.edu}\textperiodcentered\
    \phone{(+86) 136-3833-2881}\textperiodcentered\
    \github[Poncirus]{https://github.com/Poncirus}\textperiodcentered\
    \homepage[liaohanwen.com]{https://liaohanwen.com}\textperiodcentered\
    \linkedin[Hanwen Liao]{https://www.linkedin.com/in/hanwen-liao-4043b8196/}
}

% fill space
\vspace{6pt}

\section{教育经历}
\datedsubsection{\textbf{纽约大学}, 美国}{2019.09 -- (2021.01)}
专业:Computer Engineering(计算机工程)
\datedsubsection{\textbf{北京邮电大学}, 北京}{2015.09 -- 2019.06}
专业:通信工程

% fill space
\vspace{6pt}

\section{工作经历}
\datedsubsection{\textbf{猿辅导}, 北京}{2021.03 -- 现在}
软件开发工程师 - 基础研发部,直播中台,数据平台组
\begin{itemize}[parsep=0.25ex]
    \item 呈现直播中台相关大数据指标
    \item 设计抽样统计方案和协议
\end{itemize}
\datedsubsection{\textbf{华为}, 深圳}{2020.08 -- 2020.12}
软件开发工程师(实习) - 2012实验室,中央软件院,庞加莱实验室
\begin{itemize}[parsep=0.25ex]
    \item 分析C/C++语言程序的函数调用关系,了解LLVM编译原理
    \item 开发基于LLVM的变量追踪脚本
\end{itemize}
\datedsubsection{\textbf{纽约大学}, 美国}{2020.01 -- 2020.06}
助教 - 计算机结构
\begin{itemize}[parsep=0.25ex]
    \item 教授复习课,帮助学生理解课程内容
    \item 帮助学生解答作业和考试中的问题
    \item 批改作业和考试并根据情况给出反馈
\end{itemize}
\datedsubsection{\textbf{东信北邮信息技术有限公司}, 北京}{2018.06 -- 2019.08}
软件开发工程师(实习) - ISMP部门
\begin{itemize}[parsep=0.25ex]
    \item 维护ISMP(综合服务管理平台)(C++,Linux)
    \item 开发ISMP接口模块,编解码HTTP消息和JSON消息
    \item 修改Lua编译器,增加注解的语法,通过修改词法分析器禁用指定变量名
    \item 重构消息分发方式,完成负载均衡,提高可用性
    \item 开发ISMP与微服务平台的接口,传递JSON消息(Golang)
\end{itemize}

% fill space
\vspace{6pt}

\section{个人项目}
\datedsubsection{\textbf{智能健康项目}}{2018.10 -- 2019.04}
无线通信教研中心,北京邮电大学
\begin{itemize}[parsep=0.25ex]
    \item 与研究生小组一起完成智能床垫的开发
    \item 开发和维护Django服务端程序,从设备获取数据并使用MySQL数据库存储,响应客户端的请求
    \item 开发配套安卓APP,从服务端获取数据并通过Chart.js使用图表展示数据
\end{itemize}
\datedsubsection{\textbf{IoT节点设备嵌入式开发}}{2018.06 -- 2018.10}
百科融创科技发展有限公司
\begin{itemize}[parsep=0.25ex]
    \item 在STM开发板上完成IoT节点程序开发
    \item 测试多个开发板之间通过WiFi和Zigbee协议通信
    \item 开发配套安卓APP,从开发板获取数据并使用图表展示
\end{itemize}

% fill space
\vspace{6pt}

\section{技能}
\begin{itemize}[parsep=0.25ex]
    \item
          \textbf{编程语言}:
          Java, C/C++, Golang, Python, SQL, Sell Script
    \item
          \textbf{其他技术:}:
          Git, Gerrit, Linux
\end{itemize}
\end{document}
